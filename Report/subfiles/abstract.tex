{\noindent This report documents the implementation of a prototype application, which builds upon the already existing application ‘Giv et tip’. The purpose of the existing application is to crowdsource the reporting of issues, which the municipality parks department is responsible for fixing. The prototype application focuses on some of the problems of the existing application, which includes validation of reports, prioritization, and giving users an incentive to use the application.}

{\noindent A battery test was conducted on the prototype to determine the viability of the application. As the app will need precise location data continuous in the background the battery consumption might be an issue. It showed that the application will be responsible of increasing the discharge rate by 7\%.}

{\noindent The prototype solves the validation of reports by making end users validate each other's reports. The validation is done by pushing an Android notification, asking them to confirm the reported issue, whenever they pass near the geographic location of the report.}


~\\~\\
{\noindent \textit{\textbf{Keywords:} Mobile Sensing, Gamification, crowd sourcing of data, crowd sourcing of validation for data, Application development.}}