% !TEX root = ../main.tex
This section will reflect and discuss the reached solution, based upon tests and the results therefrom. It will elaborate on the parts of the problems, which was not solved. And finally reflect upon the project as a whole.

\subsection{Not Solved}
This section will address the parts of our problem, which the project has not solved, but have been planning to solve.
~\\

We wanted to create a web client, which should include a suite of administrative tools, for the municipality workers to use, to select issues to solve. In this web client, we would have solved the prioritization problem. But as we have not created this web client, as we deemed it out of the scope of the course, which focuses on mobile sensing.

Although it has not been implemented, we have made the architecture of our server application ready to serve a web client, and we believe that it would not be of great risk to implement such a client.
~\\

The lacking incentive for using the platform, has not been implemented in this prototype application, but it have been addressed, with the three solution suggestions from section X (ref 3). We have analyzed the three solutions and found both pros and cons for all implementations, and found that the lottery solution might promote cheating for personal materialistic gain, and would therefore not pursue that solution. And we found that the charity solution, was a good idea, as it could engage users, but it requires external investors in order to become a successful solution. Therefore, we find that the leaderboard solution is the easiest solution, as it do not require any external investors, and does not promote cheating by materialistic gain, but it might not be the most engaging gamification element. We therefore suggest that a combination of the three would be a great solution.
~\\

The reports in the current Giv et Tip application, has categories and images, both of which have not been implemented in this prototype, as it did not add any value to the problem domains. Although the categories could be used for weighing the prioritizations in the web client, so that more demanding categories would be weighed higher that less demanding categories.

\subsection{Project Reflection}
This section will reflect on the process of the project, the project pitch at the municipality, and the project team.
~\\

We have tried to utilize an agile approach to the development of the application, where we started by identifying the use cases, and assigned them to weekly sprints, and used a kanban system on GitHub, to keep track of the development progress. This system has worked out well for us, although we moved away from the weekly sprints, and just implemented the features, as we had time for it.
~\\

The project pitch at Odense Castle went well, this project won the “Giv et Tip” project category, and one of the judges, a developer from sweco, mentioned that we had a useful focus, with the crowdsourcing of verification.
~\\
The teamwork within the team, has worked out great, there has been great information and knowledge sharing throughout the process of this project.

\subsection{Known Issues}
In this section we will address the known issues with the current prototype, which would be fixed if there was more time for the implementation.

\begin{description}
\item [Location permissions:] There is an issue with the location service, the first time the application is started, which renders the location service useless, as it never updates, unless the application is force-stopped, through the android application settings menu. This issue is caused by the lacking confirmation of location permissions on the first startup of the application. It could probably be fixed by waiting to start the location service until it is known that location permission is granted.
\item [Shared preferences as storage:] The prototype makes excessive use of the shared preferences as data persistence. This is not optimal, and it would be much better to use SQLite, which is already included in android, for this purpose.
\item [Disabled location services:] There is another issue, which renders the location service useless. If the user disables the location services, in the android settings menu, the application will not prompt the user to re-enable it, when in the application is in focus, in order to make it usable again
\item [Download and storage of reports:] the background service is responsible for keeping report data up to date, by downloading report coordinates over time. It is currently configured to do this on a two hour interval. It would be nice to have some kind of manual update, which would offer users the option to force update the reports. Furthermore, these report coordinates is the only stored data about the reports. This means that every time the user needs to be shown report details, the application needs to download data for that report, and it is then thrown away afterwards.
\item [Settings:] the settings menu in the prototype only offers the option to tune the distance threshold to reports before a notification is shown. It could be great if there was some kinds of settings profiles in stead, which could offer e.g. a power saver mode, a  normal mode, and a power hungry mode. These profiles would be easier for an end user to understand the effects of.
\item [Geofences:] The prototype does not utilize the android api geo-fences, for figuring out which reports to calculate the distance to, but instead just calculates the distance to each and every report. This means that the application uses unnecessarily much processing power to calculate distances to reports, which are multiple kilometers away.
\end{description}

\subsection{Discussion}