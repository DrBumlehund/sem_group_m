% !TEX root = ../main.tex

This section introduces the motivation and context of the project, it describes the problem and the objective of the project, and shortly outlines the rest of the report.


\subsection{Motivation and Context}
This project has covered case number three, which works with the application 'Giv et Tip' \cite{GivEtTip}. The application is used by citizens of Odense Municipality, to report issues around the city, which the parks department is responsible for fixing. These issues include everything from potholes, to overfilled trash cans.
~\\

The objective for this case is to use mobile sensing and gamification to foster crowd sourcing of information, and keep people returning to the platform.
Mobile sensing is used in the existing application, to add extra data to the reports, such as a GPS location, to give the reports a precise geographic location, and to add images to the reports from the phone camera.

Gamification is needed in the current application, because there is no incentive to return to the platform, and users only use the application to report issues. By having a gamification system in place, the user will gain an incentive to return to the platform. This is done by adding game-like elements, such as progression, where users increase in rank and unlocks new features or content.


\subsection{Problem and Objective}
This project has worked with three problems with the current application, firstly validation of reports, which is how the municipality validates that the report is actually something they have to react upon, and fix. 
Secondly prioritization of reports, which is how the municipality prioritize the reported issues, and in which order they fix the reported issues. 
Lastly, the current application does not have an incentive to keep users on the platform, or gain new users, other than it being a easier way of reporting issues.
~\\

The objective for this project is to produce a prototype application, which will solve the report validation problem, by crowdsourcing the validations from the end users. The project will furthermore attempt to prioritize the reports, based on activity around reports. As well as as introducing some gamification elements to the platform, in order to incentivise new users, and to keep current users active on the platform. 

\subsection{Report Outline}
The report is structured as following: \autoref{sec:probdesc} describes the problems in greater detail. Section \ref{sec:solapp} describes the solution to the problem on a high level. Section \ref{sec:analysis} will analyze the problem domain. Section \ref{sec:design} builds upon the analysis and describes the design decisions made. Section \ref{sec:impl} describes the implementation of the solution. Section \ref{sec:reflect} reflects upon the achieved results, and discuss them. Section \ref{sec:conclusion} concludes the project and specify the directions for future works. \appref{app:links} contains links for the various products of this project.